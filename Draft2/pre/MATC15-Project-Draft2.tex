\documentclass{article}

%-------Packages---------
\usepackage{amssymb,amsfonts}
\usepackage[all,arc]{xy}
% \usepackage{enumerate}
\usepackage{mathrsfs}

%--------Theorem Environments--------
%theoremstyle{plain} --- default
\newtheorem{thm}{Theorem}[section]
\newtheorem{cor}[thm]{Corollary}
\newtheorem{prop}[thm]{Proposition}
\newtheorem{lem}[thm]{Lemma}
\newtheorem{conj}[thm]{Conjecture}
\newtheorem{quest}[thm]{Question}

\theoremstyle{definition}
\newtheorem{defn}[thm]{Definition}
\newtheorem{defns}[thm]{Definitions}
\newtheorem{con}[thm]{Construction}
\newtheorem{exmp}[thm]{Example}
\newtheorem{exmps}[thm]{Examples}
\newtheorem{notn}[thm]{Notation}
\newtheorem{notns}[thm]{Notations}
\newtheorem{addm}[thm]{Addendum}
\newtheorem{exer}[thm]{Exercise}

\theoremstyle{remark}
\newtheorem{rem}[thm]{Remark}
\newtheorem{rems}[thm]{Remarks}
\newtheorem{warn}[thm]{Warning}
\newtheorem{sch}[thm]{Scholium}

\makeatletter
\let\c@equation\c@thm
\makeatother
\numberwithin{equation}{section}

\bibliographystyle{plain}

%--------Meta Data: Fill in your info------
\title{Diophantine Equations to the Power of $n$ \\ \vspace{.3in} \large{MATC15 - Project - Draft 2}}

\author{Andrew D'Amario, Kevin Santos, Dawson Brown }

\date{DEADLINES: Draft II, March 25, 2021}

\begin{document}

\begin{abstract}

This is a sample latex document with emphasis on using math mode and
equation environments.  You should use it as a template for your paper.
Some pointers are included. I took the template from REU templates here~\cite{REU}.



\end{abstract}

\maketitle

\tableofcontents

\begin{flushleft}
  {\bf Conjecture:}

  \hspace{.5in}$\displaystyle x^n=\sum^{n}_{i=1}y_i^n$ has an integer solution such that $y_i\ne x, \forall i$.

  \hspace{2in} Andrew D'Amario, February 18, 2021
\end{flushleft}

\section{Introduction}
The objective of this project is to investigate the conjecture above: whether or not we can always find at least one integer solution to equations of the form $x^n=y_0^n+\cdot\cdot\cdot+y_n^n$ given any $x$, excluding trivial solutions involving $y_i$'s$=0$ or $x$. This project will be a Type II project. \\

Some of this investigation and research will involve:
\begin{itemize}    
    \item Finding parameters and conditions for possible valid solutions
    \item Computational analysis on random integers raised to the power of $n$ and finding an integer solution to the sum.
    \item Noting differences between even and odd $n$.
    \item Identifying different families of solutions that take on a similar form.
\end{itemize}

Though this conjecture may be false, we hope to investigate as much as we can on the matter and provide some deeper research to the subject.

\bibliographystyle{plain}
\bibliography{references}
\section{References}
\begin{itemize}
    \item Drago Bajc, {\bf Power solutions of some Diophantine equations}, \\
        \textit{The Mathematical Gazette}, 97:538, 107-110 (2013). \\
        https://www.jstor.org/stable/24496765
        \subitem Mentions form of above conjecture and states that solutions have been found in some cases but not in other cases, such as $n=6$. Considers above conjecture with $x^k$ instead of $x^n$, where $(k,n)=1$ and provides a general form for these solutions. 
    \item {\bf Computing Minimal Equal Sums Of Like Powers}, \\
        http://euler.free.fr/index.htm 
        \subitem Website dedicated to finding and compiling examples and counterexamples of Euler's sums of powers conjecture, which states that if a sum of $n$ positive $kth$ powers equals one $kth$ power, then $n>=k$. Includes many resources we can look into. 
    \item {\bf BEST KNOWN SOLUTIONS}, \\
        http://euler.free.fr/records.htm
        \subitem  Extensive list of aforementioned examples and counterexamples to Euler's sums of powers conjecture.
    \item L. Jacobi, D. Madden, {\bf On $a^4 + b^4 + c^4 + d^4=(a+b+c+d)^4$}, \\
        \textit{The American Mathematical Monthly}, 115:3, 230-236 (2008). \\
        https://doi.org/10.1080/00029890.2008.11920519
        \subitem Discusses specific case of the conjecture with $n=4$. Also discusses relation of Euler's conjecture and related Diophantine equations to the topic of elliptic curves. 
    \item T. Roy and F. J. Sonia, {\bf A Direct Method To Generate Pythagorean Triples And Its Generalization To Pythagorean Quadruples And n-tuples},\\ 
    https://arxiv.org/ftp/arxiv/papers/1201/1201.2145.pdf 
        \subitem Gives methods for finding Pythagorean n-tuples, sums of $n$ squares that result in a square. Might be able to reduce some cases into one of these cases. 
    \item D. R. Heath-Brown, W. M. Lioen and H. J. J. Te Riele, {\bf On Solving the Diophantine Equation $x^3 + y^3 + z^3 = k$ on a Vector Computer},\\
        \textit{Mathematics of Computation}, 61:203, 235-244 (1993)
        \subitem Presents detailed algorithm for the $n=3$ case, might be able to apply similar principles with higher $n$ values. \\
    \item L. J. Lander, T. R. Parkin and J. L. Selfridge,
       {\bf A survey of equal sums of like powers}, \\
        \textit{Mathematics of Computation}, 21, 446-459 (1967). \\
        https://www.ams.org/journals/mcom/1967-21-099/S0025-5718-1967-0222008-0/S0025-5718-1967-0222008-0.pdf 
        \subitem Presents various solutions to powers of Diophantine equations, \\
        including the $n=4$ and $n=5$ cases of the conjecture. 
    \item J. Leech, {\bf On $A^4 + B^4 + C^4 + D^4 = E^4$}, \\ 
        \textit{Mathematical Proceedings of the Cambridge Philosophical Society}, 54(4), 554-555, (1958). \\
        doi.org/10.1017/S0305004100003091
        \subitem Brief paper outlining found solutions for the $n=4$ case and considerations that reduce the number of possible solutions that need to be checked. 
\end{itemize}












\section{Writing}  Before getting to latex comments, I'll say a few words about writing, venting from many years of hard experience. 

It is important that what you write is something you would actually like to read.  There should be no overuse of symbols.
Clusters of symbols can be barbaric.  There are subjects whose literature is festooned with sentences written almost entirely with symbols like $\exists$, $\forall$.  Don't do that. Never use a symbol for a verb.  Never start a sentence with a symbol.  

Grammar should be correct.  Never start a sentence with ``Where ...".   
Mismatches of singular and plural are excruciatingly painful, utterly abhorrent.  You cannot write
``Let ..., then ..."  That is what is called a run-on sentence. You must write ``Let ... .  Then ... ."
Alternatively, ``If ..., then ..."  works just fine.

Avoid words or phrases like Simply, Obviously, Just, ..., It is easy to see.  They serve only to intimidate or to browbeat the reader into acquiescence. 
Especially if English is not your native language, make sure you have  somebody fluent in English check what you have written.     

References to numbered statements are best in the form Theorem 2.3, not just 2.3.
Equations should be referred to as (2.3), NOT 2.3 or equation 2.3 or equation (2.3).


Bad writing makes for unpleasant papers, no matter how good the material.


\section{What does the table of contents command do?}

The table of contents command will automatically make a contents.
You must run tex at least twice for this to work.

\section{How do the environment commands work?}

``Environments'' are commands that are given using the \verb|\begin{}|
  and \verb|\end{}| syntax. In the preamble, you can see we've defined
  a bunch of theorem-type environments.  For example, to get a definition, 
you type:

\begin{defn}  This is how to define a definition.
\end{defn}

And for a theorem and its proof you type:

\begin{thm}
This is the statement of a theorem.
\end{thm}
\begin{proof}
And this shows that the statement is correct.
\end{proof}

Note that the numbering is taken care of automatically, and that we've predefined a bunch of these sorts of environments in the header.  These take give environments for  lemmas, corollaries and such like. 

Another useful kind of enviroment is the equation environment.  Equations
get numbered in sequence with statements, as for example

\begin{equation}  e = mc^2
\end{equation}

Note that if you do not want a numbered equation, you can use the
environment ``equation*''
 like so:

\begin{equation*}
e=mc^2
\end{equation*}
But a quicker way to tex the same command and get the same displayed result is:
\[  e=mc^2 \]

There are plenty of other equation-type enviroments that allow you to
align several equations and such like things. The AMS's guide \cite{amsshort} is a
good place to start with these.

You can also typeset math directly in a paragraph by placing it within
dollar signs.  This is called ``math mode.''  For example: Let $e$
be energy, $m$ be momentum and $c$ be the speed of light.  Then
Einstein's famous equation says that $e=mc^2$.  This is useful, but
remember that it is harder to read inline math than displayed math. 

Remember that letters get put in a different font in math mode, so
whenever you are referencing a mathematical object you should always
put it in dollar signs.  For example, $f$ is a function, but f is just
a random letter.

Both \cite{notsoshort} and \cite{amsshort} have good lists of other symbols you can use in math mode.
These include greek letters ($\alpha, \beta, \Gamma, \Delta$),
operators ($\otimes, +, \sum$) and much more ($\leq, \diamond$).

\section{Xypic and diagrams}

If you want to draw diagrams, you should use xypic.  It's actually
much easier than it looks, and we've already included it in the header
above.  Here is an example. 
\[\xymatrix{
FX \ar[r]^-{Ff} \ar[d]_{\eta_X} & FY \ar[d]^{\eta_Y} \\
GX \ar[r]_-{Gf} & GY\\} \]

\section*{Acknowledgments}  You should thank anyone who deserves thanks, and for sure you should
thank your mentor.   ``It is a pleasure to thank my mentor, 
his/her name, for ....  ".   Or add anyone else, for example ``I thank [another participant] for helping 
me understand [something or other]"

\section{bibliography}  The bibliography should list all sources that you have used and referenced.
And you should reference anything you use.   Especially if you quote any result without proof, you MUST
give a reference.   And never ever should you copy material directly or more or less directly, from a source.

\begin{thebibliography}{9}
\bibitem{REU} http://math.uchicago.edu/~may/REU2020/
\bibitem{ams} http://www.ams.org/publications/authors/tex/amslatex

\bibitem{amsshort}
Michael Downes.
Short Math Guide for \LaTeX.
http://tex.loria.fr/general/downes-short-math-guide.pdf

\bibitem{May}
J. P. May.
A Concise Course in Algebraic Topology.
University of Chicago Press. 1999. 

\bibitem{notsoshort}
Tobias Oekiter, Hubert Partl, Irene Hyna and Elisabeth Schlegl.
The Not So Short Introduction to \LaTeX 2e.
https://tobi.oetiker.ch/lshort/lshort.pdf

\end{thebibliography}

\end{document}

